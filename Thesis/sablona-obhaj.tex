% Soubory musí být v kódování, které je nastaveno v příkazu \usepackage[...]{inputenc}

\documentclass[%        Základní nastavení
  %draft,    				  % Testovací překlad
  12pt,       				% Velikost základního písma je 12 bodů
	t,                  % obsah slajdů bude vždy začínat od shora (nebude vertikálně centrovaný)
	aspectratio=1610,   % poměr stran bude 16:10 (všechny projektory v učebnách na Technické 12 Brno),
	                    % další volby jsou 43, 149, 169, 54, 32.
	unicode,						% Záložky a informace budou v kódování unicode
]{beamer}				    	% Dokument třídy 'zpráva', vhodná pro sazbu závěrečných prací s kapitolami
%\usepackage{etex}

\usepackage[utf8]		  % Kódování zdrojových souborů je v UTF-8
	{inputenc}					% Balíček pro nastavení kódování zdrojových souborů
	
\usepackage{graphicx} % Balíček 'graphicx' pro vkládání obrázků
											% Nutné pro vložení logotypů školy a fakulty

\usepackage[          % Balíček 'acronym' pro sazby zkratek a symbolů
	nohyperlinks				% Nebudou tvořeny hypertextové odkazy do seznamu zkratek
]{acronym}						
											% Nutné pro použití prostředí 'acronym' balíčku 'thesis'

%% Balíček hyperref je volán třídou beamer automaticky, proto není třeba následujícího kódu:
%\usepackage[
%	breaklinks=true,		% Hypertextové odkazy mohou obsahovat zalomení řádku
%	hypertexnames=false % Názvy hypertextových odkazů budou tvořeny
%											% nezávisle na názvech TeXu
%]{hyperref}						% Balíček 'hyperref' pro sazbu hypertextových odkazů
%											% Nutné pro použití příkazu 'nastavenipdf' balíčku 'thesis'

\usepackage{cmap} 		% Balíček cmap zajišťuje, že PDF vytvořené `pdflatexem' je
											% plně "prohledávatelné" a "kopírovatelné"

%\usepackage{upgreek}	% Balíček pro sazbu stojatých řeckých písmem
											%% např. stojaté pí: \uppi
											%% např. stojaté mí: \upmu (použitelné třeba v mikrometrech)
											%% pozor, grafická nekompatibilita s fonty typu Computer Modern!

%\usepackage{amsmath} %balíček pro sabu náročnější matematiky

\usepackage{booktabs} % Balíček, který umožňuje v tabulce používat
                      % příkazy \toprule, \midrule, \bottomrule

%%%%%vlastni balicky
\usepackage{makecell}
\usepackage{siunitx}
\sisetup{output-decimal-marker = {,}}
\DeclareSIUnit\LSB{LSB}
\sisetup{
  inter-unit-product=\ensuremath{{\cdot}},
%  tight-spacing=true,
}
\usepackage{subfigure}
\usepackage{tablefootnote}
\usepackage{amsmath}
\usepackage[thinc]{esdiff}


%%%%%%%%%%%%%%%%%%%%%%%%%%%%%%%%%%%%%%%%%%%%%%%%%%%%%%%%%%%%%%%%%
%%%%%%      Definice informací o dokumentu             %%%%%%%%%%
%%%%%%%%%%%%%%%%%%%%%%%%%%%%%%%%%%%%%%%%%%%%%%%%%%%%%%%%%%%%%%%%%

% V tomto souboru se nastavují téměř veškeré informace, proměnné mezi studenty:
% jméno, název práce, pohlaví atd.
% Tento soubor je SDÍLENÝ mezi textem práce a prezentací k obhajobě -- netřeba něco nastavovat na dvou místech.

\usepackage[
%%% Z následujících voleb jazyka lze použít pouze jednu
  czech-english,		% originální jazyk je čeština, překlad je anglicky (výchozí)
  %english-czech,	% originální jazyk je angličtina, překlad je česky
  %slovak-english,	% originální jazyk je slovenština, překlad je anglicky
  %english-slovak,	% originální jazyk je angličtina, překlad je slovensky
%
%%% Z následujících voleb typu práce lze použít pouze jednu
  %semestral,		  % semestrální práce (výchozí)
  bachelor,			%	bakalářská práce
  %master,			  % diplomová práce
  %treatise,			% pojednání o disertační práci
  %doctoral,			% disertační práce
%
%%% Z následujících voleb zarovnání objektů lze použít pouze jednu
%  left,				  % rovnice a popisky plovoucích objektů budou zarovnány vlevo
	center,			    % rovnice a popisky plovoucích objektů budou zarovnány na střed (vychozi)
%
]{thesis}   % Balíček pro sazbu studentských prací


%%% Jméno a příjmení autora ve tvaru
%  [tituly před jménem]{Křestní}{Příjmení}[tituly za jménem]
% Pokud osoba nemá titul před/za jménem, smažte celý řetězec '[...]'
\author{Marek}{Coufal}

%%% Identifikační číslo autora (VUT ID)
\butid{240598}

%%% Pohlaví autora/autorky
% (nepoužije se ve variantě english-czech ani english-slovak)
% Číselná hodnota: 1...žena, 0...muž
\gender{0}

%%% Jméno a příjmení vedoucího/školitele včetně titulů
%  [tituly před jménem]{Křestní}{Příjmení}[tituly za jménem]
% Pokud osoba nemá titul před/za jménem, smažte celý řetězec '[...]'
\advisor[Ing.]{Jan}{Král}[Ph.D.]

%%% Jméno a příjmení oponenta včetně titulů
%  [tituly před jménem]{Křestní}{Příjmení}[tituly za jménem]
% Pokud osoba nemá titul před/za jménem, smažte celý řetězec '[...]'
% Nastavení oponenta se uplatní pouze v prezentaci k obhajobě;
% v případě, že nechcete, aby se na titulním snímku prezentace zobrazoval oponent, pouze příkaz zakomentujte;
% u obhajoby semestrální práce se oponent nezobrazuje (jelikož neexistuje)
% U dizertační práce jsou typicky dva až tři oponenti. Pokud je chcete mít na titulním slajdu, prosím ručně odkomentujte a upravte jejich jména v definici "VUT title page" v souboru thesis.sty.
\opponent[Ing.]{Josef}{Vychodil}[Ph.D.]

%%% Název práce
%  Parametr ve složených závorkách {} je název v originálním jazyce,
%  parametr v hranatých závorkách [] je překlad (podle toho jaký je originální jazyk).
%  V případě, že název Vaší práce je dlouhý a nevleze se celý do zápatí prezentace, použijte příkaz
%  \def\insertshorttitle{Zkác.\ náz.\ práce}
%  kde jako parametr vyplníte zkrácený název. Pokud nechcete zkracovat název, budete muset předefinovat,
%  jak se vytváří patička slidu. Viz odkaz: https://bit.ly/3EJTp5A
\title[Indoor Positioning Based on Inercial Measurement Unit
]{Měření polohy uvnitř budov pomocí inerciální jednotky
}

%%% Označení oboru studia
%  Parametr ve složených závorkách {} je název oboru v originálním jazyce,
%  parametr v hranatých závorkách [] je překlad
\specialization[Electronics and Communication Technologies]{Elektronika a komunikační technologie}

%%% Označení ústavu
%  Parametr ve složených závorkách {} je název ústavu v originálním jazyce,
%  parametr v hranatých závorkách [] je překlad
%\department[Department of Control and Instrumentation]{Ústav automatizace a měřicí techniky}
%\department[Department of Biomedical Engineering]{Ústav biomedicínského inženýrství}
%\department[Department of Electrical Power Engineering]{Ústav elektroenergetiky}
%\department[Department of Electrical and Electronic Technology]{Ústav elektrotechnologie}
%\department[Department of Physics]{Ústav fyziky}
%\department[Department of Foreign Languages]{Ústav jazyků}
%\department[Department of Mathematics]{Ústav matematiky}
%\department[Department of Microelectronics]{Ústav mikroelektroniky}
\department[Department of Radio Electronics]{Ústav radioelektroniky}
%\department[Department of Theoretical and Experimental Electrical Engineering]{Ústav teoretické a experimentální elektrotechniky}
%\department[Department of Telecommunications]{Ústav telekomunikací}
%\department[Department of Power Electrical and Electronic Engineering]{Ústav výkonové elektrotechniky a elektroniky}

%%% Označení fakulty
%  Parametr ve složených závorkách {} je název fakulty v originálním jazyce,
%  parametr v hranatých závorkách [] je překlad
%\faculty[Faculty of Architecture]{Fakulta architektury}
\faculty[Faculty of Electrical Engineering and~Communication]{Fakulta elektrotechniky a~komunikačních technologií}
%\faculty[Faculty of Chemistry]{Fakulta chemická}
%\faculty[Faculty of Information Technology]{Fakulta informačních technologií}
%\faculty[Faculty of Business and Management]{Fakulta podnikatelská}
%\faculty[Faculty of Civil Engineering]{Fakulta stavební}
%\faculty[Faculty of Mechanical Engineering]{Fakulta strojního inženýrství}
%\faculty[Faculty of Fine Arts]{Fakulta výtvarných umění}
%
%Nastavení logotypu (v hranatych zavorkach zkracene logo, ve slozenych plne):
\facultylogo[logo/FEKT_zkratka_barevne_PANTONE_CZ]{logo/VUT_barevne_PANTONE_CZ}

%%% Rok odevzdání práce
\graduateyear{2024}
%%% Akademický rok odevzdání práce
\academicyear{2023/24}

%%% Datum obhajoby (uplatní se pouze v prezentaci k obhajobě)
\date{11.\,6.\,2024} 

%%% Místo obhajoby
% Na titulních stránkách bude automaticky vysázeno VELKÝMI písmeny (pokud tyto stránky sází šablona)
\city{Brno}

%%% Abstrakt
\abstract[%
This bachelor's thesis deals with inertial navigation intended for indoor usage. The algorithm for processing accelerometer and gyro data is described along with their position in equations of motion. Various coordinate systems are defined for ease of manipulation of vector quantities, including conversions and rotations between frames. The thesis also addresses the design of the hardware for the inertial unit, including the minimum requirements for fast data recording. Some chapters are dedicated to the design of the printed circuit board and the implementation of the device, followed by the development of firmware and software for data logging and processing.
]{%
Tato bakalářská práce se zabývá inerciální navigací pro použití uvnitř budov. Je rozebrán algoritmus zpracování dat akcelerometrů a gyroskopů a jejich zpracování pomocí pohybových rovnic. Jsou také definovány různé vztažné soustavy pro potřeby manipulace s vektorovými fyzikálními veličinami a jejich převody a rotace mezi soustavami. Práce se také věnuje návrhu hardwaru inerciální jednotky, minimálním požadavkům pro rychlý záznam dat. Jsou věnovány kapitoly samotnému návrhu desky plošných spojů a realizaci zařízení, následnému vývoji firmwaru a softwaru na přenos a zpracování dat.
}

%%% Klíčová slova
\keywrds[%
IMU, inertial navigation, position, orientation, strapdown, coordinate systems, Euler, gravity, gyro, accelerometer, magnetometer, GNSS, STM32, ESP32, MATLAB, calibration
]{%
IMU, inerciální navigace, poloha, orientace, připevněný, vztažné soustavy, Euler, gravitace, gyroskop, akcelerometr, magnetometr, GNSS, STM32, ESP32, MATLAB, kalibrace
}

%%% Poděkování
\acknowledgement{%
Rád bych poděkoval vedoucímu bakalářské práce
panu Ing.~Janu Královi, Ph.D.\ za odborné vedení,
konzultace, trpělivost, vstřícnost a~podnětné návrhy k~práci.
}%      % v tomto souboru doplňte údaje o sobě, o názvu práce...
                       % (tento soubor je sdílený s textem práce)

%%%%%%%%%%%%%%%%%%%%%%%%%%%%%%%%%%%%%%%%%%%%%%%%%%%%%%%%%%%%%%%%%%%%%%%%

%%%%%%%%%%%%%%%%%%%%%%%%%%%%%%%%%%%%%%%%%%%%%%%%%%%%%%%%%%%%%%%%%%%%%%%%
%%%%%%     Nastavení polí ve Vlastnostech dokumentu PDF      %%%%%%%%%%%
%%%%%%%%%%%%%%%%%%%%%%%%%%%%%%%%%%%%%%%%%%%%%%%%%%%%%%%%%%%%%%%%%%%%%%%%
%% Při vloženém balíčku 'hyperref' lze použít příkaz '\pdfsettings'
\pdfsettings
%  Nastavení polí je možné provést také ručně příkazem:
%\hypersetup{
%  pdftitle={Název studentské práce},    	% Pole 'Document Title'
%  pdfauthor={Autor studenstké práce},   	% Pole 'Author'
%  pdfsubject={Typ práce}, 						  	% Pole 'Subject'
%  pdfkeywords={Klíčová slova}           	% Pole 'Keywords'
%}
\hypersetup{pdfpagemode=FullScreen}       % otevření rovnou v režimu celé obrazovky
%%%%%%%%%%%%%%%%%%%%%%%%%%%%%%%%%%%%%%%%%%%%%%%%%%%%%%%%%%%%%%%%%%%%%%%

\usetheme{VUT} 				% barvy a rozložení prezentace odpovídající VUT FEKT
% alternativně lze použít jiná berevná témata, ale bez záruky. Například: 
%\usetheme{Darmstadt} \usecolortheme{default2}
\logoheader					% vytvoření zkráceného loga VUT FEKT v hlavičce slajdu, nechte odkomentované



\begin{document}

% v případě zakomentování následujícího se zobrazí v pravém dolním rohu slajdů klikatelné navigační symboly 
\disablenavigationsymbols

% titulní snímek, vysazen bez horních, dolních a postranních lišt (volba plain),
% není tak vysazen ani nadpis snímku
\maketitle

%%%%%%%%%%%%%%%%%%%%%%%%%%%%%%%%%%%%%%%%%%%%%%%%%%%%%%%%%%%%%%%%%%%%%%%
% 1. snímek s cíli (zadaním) práce
\begin{frame} 
	% nadpis snímku
	\frametitle{Cíle práce}
	\begin{itemize}
			\item Nastudovat
			\begin{itemize}
					\item dostupné inerciální jednotky
				\end{itemize}
			\item Hardware
				\begin{itemize}
					\item vybrat vhodné senzory a komponenty bezdrátové jednotky
					\item analyzovat potřeby záznamu dat a jejich ukládání
					\item návrh DPS a realizace
				\end{itemize}

	\end{itemize}
\end{frame}

%%%%%%%%%%%%%
\begin{frame} 
	\frametitle{Princip fungování inerciální navigace}
	
			\begin{figure}%	
				\centering
				
				\includegraphics[width=0.8\columnwidth]{obrazky/StrapdownBlock}
				%lze vložit popisek, ale povetšinou je to v prezentaci zbytečné
				\caption{Blokové schéma algoritmu strapdown inerciální navigace, převzato z [1] [2]}%
				%\label{obr:ukazka}
			\end{figure}
	
\end{frame} 


%%%%%%%%%%%%%
\begin{frame} 
	\frametitle{Fúze dat z jiných senzorů}
		\begin{alertblock}{Nepřesnost}
		S časem díky integraci roste chyba měření.
		\end{alertblock}
		\vspace{4ex}
		\begin{block}{Možnosti snížení chyby}
		\begin{itemize}
		\item GNSS - při částečně dostupném signálu
		\item Magnetometr - omezení gyro driftu v horizontální rovině
		\end{itemize}
		\end{block}

	
\end{frame} 

%%%%%%%%%%%%%
\begin{frame} 
	\frametitle{Testování vývojových kitů}
	
	\begin{columns}[T] 								% prostředí sloupce s umístěním nahoře
		\begin{column}{0.3\textwidth}		% první sloupec
			3D tištěný držák pro zarovnání geometrických os:
			%
			\begin{itemize}
				\item IMU - ADIS16505
				\item GNSS - NEO-M8U
			\end{itemize}
			\vspace{0.5cm}
			Zpracování dat:
			%
			\begin{itemize}
				\item Matlab navigation toolbox - převážně pouze pro natočení, ne polohad
				\item Asynchronost USB komunikace
			\end{itemize}
		\end{column}
		%
		\begin{column}{0.7\textwidth}		% druhý sloupec
			\begin{figure}%	
				\centering
				\includegraphics[width=0.9\columnwidth]{obrazky/devBoards}
				%lze vložit popisek, ale povetšinou je to v prezentaci zbytečné
				\caption{Testovací přípravek s vývojovými deskami}%
				%\label{obr:ukazka}
			\end{figure}
		\end{column}
	\end{columns}						
			

	
\end{frame} 

%%%%%%%%%%%%%
\begin{frame} 
	\frametitle{Hardware inerciální jednotky}
	
			\begin{figure}%	
				\centering
				
				\includegraphics[width=0.55\columnwidth]{obrazky/IMUnav_H00_block}
				%lze vložit popisek, ale povetšinou je to v prezentaci zbytečné
				\caption{Blokové schéma inerciální jednotky}%
				%\label{obr:ukazka}
			\end{figure}
	
\end{frame} 

%%%%%%%%%%%%%
\begin{frame} 
	\frametitle{Požadavky na sběrnice}
	\begin{table}[ht]
\centering
\begin{tabular}{|c|c|c|}
\hline 
Druh periferie & Minimální požadovaný počet & Použití periferie \\ 
\hline 
\hline 
I2C & 3 & \makecell{OLED displej, LSM303AGR, \\MPU6050, USB2514B}  \\ 
\hline 
SPI & 1 & ADIS16505 \\ 
\hline 
UART & 2 & NEO-M8U, ESP32 \\ 
\hline 
QUADSPI & 1 & NOR FLASH paměť \\ 
\hline 
SDIO & 1 & microSD karta \\ 
\hline 
ADC & 1 & měření napětí akumulátoru \\ 
\hline 
\end{tabular} 

\caption{Minimální požadavky na periferie mikroprocesoru.} 
\label{table:MCUperiferie}
\end{table} 

	

	
\end{frame} 


%%%%%%%%%%%%%
\begin{frame} 
	\frametitle{Požadavky na paměť}
	\begin{table}[h!]
\caption{Odhad celkového datového toku pro záznam dat.} 
\centering
\begin{tabular}{c|c}

Senzor & Odhadovaný datový tok \\ 
\hline 
\hline 
ADIS16505-2 & 375 kbit/s \\ 

MPU-6050 & 422 kbit/s \\ 

LSM303AGR & 7 kbit/s \\ 

NEO-M8U & 1 kbit/s \\ 
\hline

Celkem & 805 kbit/s (0,1MB/s) \\ 

\end{tabular} 

\label{table:memoryBW}
\end{table} 
	
	12 MB dat při dvouminutovém záznamu.
	
	\begin{itemize}
	\item SD karta
	\item 32 MB NOR Flash
	\end{itemize}
	
\end{frame} 


%%%%%%%%%%%%%
\begin{frame} 
	\frametitle{Plošný spoj}
	
	\begin{columns}[T] 								% prostředí sloupce s umístěním nahoře
		\begin{column}{0.3\textwidth}		% první sloupec
			Obrázek znázorňuje model:\\[2ex]
			%
			\begin{itemize}
				\item Deska
				\item Součástky
				\item Signály
				\item Napájení
			\end{itemize}
		\end{column}
		%
		\begin{column}{0.7\textwidth}		% druhý sloupec
			\begin{figure}%	
				\centering
			    \includegraphics[width=0.6\columnwidth, trim={0 3.2cm 0 1.5cm},clip]{KiCad/boardTopParts}
				%lze vložit popisek, ale povetšinou je to v prezentaci zbytečné
				\caption{Pohled osazení součástek}%
				%\label{obr:ukazka}
			\end{figure}
		\end{column}
	\end{columns}											% ukončení prostředí sloupce
\end{frame}





\begin{frame} 
	\frametitle{3D model desky}
	\begin{figure}%	
				\centering
				
				\includegraphics[width=0.5\columnwidth]{KiCad/3Dboard}
				%lze vložit popisek, ale povetšinou je to v prezentaci zbytečné
				\caption{3D model navržené DPS}%
				%\label{obr:ukazka}
			\end{figure}
\end{frame}


%%%%%%%%%%%%%
\begin{frame} 
	\frametitle{Cíle pro bakalářskou práci}
	
	\begin{itemize}
	\item Firmware inerciální jednotky
	\item Navrhnout komunikaci mezi PC a jednotkou + obslužný SW
	\item Software na zpracování dat (Matlab / Python)
	\item Možnosti fúze s GNSS, magnetometrem
	\item Využití v rámci MPC-RAR
	\end{itemize}
\end{frame}


% podekovani
\begin{frame}[c] 
% bez nadpisu snímku
	\frametitle{\mbox{ }}
	\begin{center}
		{\Huge Děkuji za pozornost!}
	\end{center}
\end{frame}

% zdroje
\begin{frame}[c] 
\frametitle{Zdroje}
[1] TITTERTON, D. H. a WESTON, J. L. \textit{Strapdown inertial navigation technology}. Second edition. Progress in astronautics and aeronautics, 207. Reston, VA: American Institute of Aeronautics and Astronautics, c2004. ISBN 1-56347-693-2.

[2] GREWAL, Mohinder S.; ANDREWS, Angus P. a BARTONE, Chris. \textit{Global navigation satellite systems, inertial navigation, and integration}. Third edition. Hoboken, New Jersey: John Wiley, 2013. ISBN 978-1-118-44700-0.
\end{frame}


\end{document}
