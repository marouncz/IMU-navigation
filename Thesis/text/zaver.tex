\chapter*{Závěr}
\phantomsection
\addcontentsline{toc}{chapter}{Závěr}
V rámci bakalářské práce byla popsána kinematika pohybu a nakládání s veličinami změřenými IMU pro potřeby výpočtu polohy. Také bylo definováno několik vztažných soustav a postupy pro převod mezi nimi. Je rozebráno tíhové pole Země, gravitační modely a jejich význam v inerciální navigaci.

Byl popsán funkční princip IMU a společně s GNSS modulem s možností inerciální navigace byly vyzkoušeny a otestovány pomocí běžně dostupných vývojových stavebnic.

Práce se také věnuje návrhu obvodového zapojení inerciální jednotky, definováním minimálních požadavků na hlavní MCU tak, abychom nebyli v budoucnu omezeni některým z rozhodnutí při návrhu hardwaru. Inerciální jednotka byla osazena i jinými senzory než gyroskopy a akcelerometry pro možnou senzorickou fúzi. Dále byl popsán návrh samotné DPS v programu KiCad, jejíž výkresy a schéma jsou v příloze. Je popsána konstrukce celého zařízení, byla vytvořena 3D tištěná krabička a postup oživení.

Další část práce se věnuje vývoji firmwaru pro MCU a použitým nástrojům. Funkční bloky firmwaru byly rozloženy do několika stavových automatů pro větší přehlednost a jejich funkcionalita popsána. Také byl vytvořen skript pro převod binárních dat do jednoduše čitelného souborového formátu.
Předposlední kapitola práce je věnována vytvořeným skriptům na zpracování dat z čistě inerciálních senzorů a jeho omezením a vlivu kalibrace na výsledek. Také byla otestována fúze dat z GNSS modulu, u které je pravděpodobně možné dosáhnout řádově lepších výsledků navazujícím výzkumem. V elektronické příloze jsou také vzorová data z různých měření pro případ, že by se někdo chtěl zabývat jejich zpracováním bez toho, aby měl zařízení fyzicky u sebe.

V poslední kapitole jsou krátce shrnuty dvě možné koncepce využití zařízení pro laboratorní úlohu, ovšem vzhledem k univerzálnosti vytvořeného zařízení se můžou najít i jiné oblasti jeho využití.

