\chapter*{Závěr}
\phantomsection
\addcontentsline{toc}{chapter}{Závěr}
V rámci semestrální práce jsme popsali kinematiku pohybu a nakládání s veličinami změřenými IMU pro potřeby výpočtu polohy. Také jsme definovali několik vztažných soustav a postupy pro převod mezi nimi. Je rozebráno tíhové pole Země, gravitační modely a jejich význam v inerciální navigaci.

Byl popsán funkční princip IMU a společně s GNSS modulem s možností inerciální navigace byly vyzkoušeny a otestovány pomocí běžně dostupných vývojových stavebnic.

Práce se také věnuje návrhu obvodového zapojení inerciální jednotky, definováním minimálních požadavků na hlavní MCU tak, abychom nebyli v budoucnu omezeni některým z rozhodnutí při návrhu hardwaru. Inerciální jednotka byla osazena i jinými senzory než gyroskopy a akcelerometry pro možnou senzorickou fúzi v bakalářské práci.

V neposlední řadě se také věnujeme návrhu samotné DPS v programu KiCad, jejíž výkresy a schéma jsou v příloze. Deska je v čase psaní této práce buď vyráběna, nebo již čeká na doručení.