\chapter{Firmware inerciální jednotky}
Firmware hlavního \ac{MCU} byl vyvíjen pomocí volně dostupného \ac{IDE} poskytovaného výrobcem - \emph{STM32CubeIDE}. Jedná se o nástroj určený pro práci s jazyky C/C++, GCC kompilátorem založeném na Eclipse \cite{V2Tf5wsrbWcbQdoW}. Zároveň poskytuje grafické rozhraní pro konfiguraci a generování knihoven založených na \ac{HAL}, možnost použití \ac{RTOS} a ladící prostředí.

V této práci byly použity poskytované knihovny HAL a FreeRTOS pro ulehčení a urychlení vývoje firmwaru. Jejich použití často s sebou nese nevýhody, jako je například horší využití paměti, nebo výpočetního výkonu, z tohoto důvodu byl zvolen takový \ac{MCU}, aby měl dostatečné rezervy pro jejich použití.
\section{HAL}
Generování kódu \ac{HAL} v  STM32CubeIDE je možné pomocí grafického rozhraní, které poskytuje uživateli možnost nastavení jednotlivých pinů, periferií, komunikačních rozhraní a vnitřních hodin (obrázek \ref{fig:cubeConfig}). Vygenerované knihovny následně umožňují uživateli pracovat s \ac{MCU} s jistou mírou abstrakce, například není nutné znát a pracovat s názvy jednotlivých registrů. Typickým příkladem můžou být komunikační sběrnice (\ac{SPI}, \ac{I2C} \ldots), pro které jsou dostupné obslužné funkce na čtení a vysílání dat, jak v blokujícím režimu, tak i v neblokujícím (například pomocí \ac{DMA}). \cite{V2Tf5wsrbWcbQdoW}

Dále je možné pomocí stejného grafického rozhraní importovat rozšiřující softwarové balíčky, i když už se nejedná přímo o \ac{HAL}. V této práci byly použity \emph{FATFS} pro manipulaci se soubory na microSD kartě, \emph{FreeRTOS} jakožto jeden z dostupných \ac{RTOS} a \emph{USB\_DEVICE} pro práci s \ac{USB} rozhraním třídy \ac{MSC}.

\begin{figure}[h]
     \centering
     \begin{subfigure}[b]{0.4\textwidth}
         \centering
         \includegraphics[width=\textwidth]{obrazky/cubePinout}
         \caption{Konfigurace pinů}
       
     \end{subfigure}
     \hfill
     \begin{subfigure}[b]{0.4\textwidth}
         \centering
         \includegraphics[width=\textwidth]{obrazky/cubeClock}
         \caption{Konfigurace hodin}
         
     \end{subfigure}
        \caption{Konfigurace MCU v STM32CubeIDE}
        \label{fig:cubeConfig}
\end{figure}

\section{FreeRTOS}
V této aplikaci je potřeba vyčítat, převádět a zapisovat data z několika různých senzorů které nemají přesně stejný hodinový signál, zároveň obsluhovat \ac{GUI} a provádět záznam dat zároveň. Pro potřeby synchronizace několika úloh, které nemají stejné periody, nebo například čekají na vstup od uživatele se hodí \ac{RTOS}.

Byla vybrána jedna z variant operačních systémů reálného času, a to FreeRTOS. Jedná se o jednoduchý open-source systém, který je hojně využíván ve vestavěných aplikacích. Umožňuje aplikaci virtuálně rozdělit na několik samostatných vláken (tzv. \emph{tasků}) s různými prioritami. Časování tasků je možné například pomocí neblokujících prodlev, nebo semaforů. FreeRTOS také plní funkci správy a alokace paměti. Předávání informací mezi jednotlivými tasky se provádí pomocí tzv. \emph{Queues}, díky tomu je možné se vyvarovat použití globálních proměnných. \cite{Zhu2011}

Práce s FreeRTOS je v STM32CubeIDE zjednodušená také díky poměrně dobré možnosti ladit aplikace pomocí již vestavěného RTOS-aware debuggeru, díky kterému můžeme například analyzovat využití paměti jednotlivých tasků, využití času, nebo kontrolovat stavy semaforů a velikost obsazených Queues. 

\section{Vývojové diagramy firmwaru}
Popsat chování a funkcionalitu firmwaru této aplikace dohromady by bylo poměrně nepřehledné. Proto budou jednotlivé funkce rozděleny do několika samostatných logických bloků, kde každý blok reprezentuje jeden task operačního systému.

\subsection{KeepaliveTask}
\begin{figure}[h]
    \centering
    \includegraphics[width=0.3\textwidth]{obrazky/KeepaliveTask}
    \caption{Vývojový diagram KeepaliveTask}
\end{figure}
Jedná se o úlohu s nastavenou nejnižší prioritou. Slouží pouze pro ladicí účely a umožňuje jednoduchou a rychlou reprezentaci stavu systému pomocí blikající LED, zdali je spouštěn i task s nejnižší prioritou.
\subsection{hubTask}
\begin{figure}[h]
    \centering
    \includegraphics[width=0.25\textwidth]{obrazky/HubTask}
    \caption{Vývojový diagram HubTask}
\end{figure}
Tato úloha vykonává funkce pouze při zapnutí zařízení, a to konfiguraci a sepnutí vestavěného \ac{USB} rozbočovače. Do něj jsou nahrána konfigurační data pomocí sběrnice I2C, jako je například \ac{VID}, \ac{PID} a nastavení jednotlivých portů. Následně jsou registry rozbočovače přepnuty do režimu pouze pro čtení a \ac{USB} rozhraní zapnuto.
\subsection{powerTask}
V tomto tasku jsou periodicky měřena všechna analogová napětí pomocí \ac{ADC} procesoru, například napětí zdroje, USB portu, akumulátoru, ale i teplota procesoru. Tyto stavové veličiny jsou zobrazovány pomocí \ac{GUI}. Čtené hodnoty napětí akumulátoru jsou průměrovány pomocí pohyblivého exponenciálního filtru, který je možný zapsat pomocí rovnice \ref{eq:EMA}.

\begin{equation} \label{eq:EMA}
y[n]=\alpha \cdot x[n] + (1-\alpha)\cdot y[n-1]
\end{equation}

Kde $ x[n] $ je přečtená hodnota napětí, $ y[n] $ vyfiltrovaná hodnota napětí, $ y[n-1] $ výsledek vyfiltrované hodnoty napětí z předešlého cyklu a $ \alpha $ je nastavitelný koeficient odezvy filtru. Experimentálně bylo odzkoušeno, že vhodných výsledků filtrace šumu je možné dosáhnout s $ \alpha = 0,3$.

V případě, že klesne napětí akumulátoru pod hranici 3,5 V je zařízení vypnuto překlopení S/R klopného obvodu, který je zmiňovaný v kapitole \ref{hardware} a tím je dosažena ochrana akumulátoru proti podvybití.


\begin{figure}[h]
    \centering
    \includegraphics[width=0.3\textwidth]{obrazky/PowerTask}
    \caption{Vývojový diagram PowerTask}
\end{figure}
\subsection{gpsTask}
\begin{figure}[h]
    \centering
    \includegraphics[width=0.3\textwidth]{obrazky/GpsTask}
    \caption{Vývojový diagram GpsTask}
\end{figure}
\subsection{lsmTask}
\begin{figure}[h]
    \centering
    \includegraphics[width=0.6\textwidth]{obrazky/LsmTask}
    \caption{Vývojový diagram LsmTask}
\end{figure}
\subsection{mpuTask}
\begin{figure}[h]
    \centering
    \includegraphics[width=0.6\textwidth]{obrazky/MpuTask}
    \caption{Vývojový diagram MpuTask}
\end{figure}
\subsection{adisTask}
\begin{figure}[h]
    \centering
    \includegraphics[width=0.95\textwidth]{obrazky/AdisTask}
    \caption{Vývojový diagram AdisTask}
\end{figure}
\subsection{loggerTask}
\begin{figure}[h]
    \centering
    \includegraphics[width=0.95\textwidth]{obrazky/LoggerTask}
    \caption{Vývojový diagram LoggerTask}
\end{figure}
\subsection{oledTask}
\begin{figure}[h]
    \centering
    \includegraphics[width=0.95\textwidth]{obrazky/OledTask}
    \caption{Vývojový diagram OledTask}
\end{figure}

...jednotlivé sekce a subsekce vývojových diagramů, hlavní vývojový diagram

gui

kalibrace

testy

zarovnávání bytů v paměti



