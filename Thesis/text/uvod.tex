\chapter*{Úvod}
\phantomsection
\addcontentsline{toc}{chapter}{Úvod}

Tato práce se zabývá poměrně komplexní problematikou inerciální navigace pro použití uvnitř budov a jejím sestrojením, tedy určováním polohy pomocí „black boxu“ který své údaje o pohybu určí primárně z údajů o lineárním zrychlení z akcelerometru a úhlové rychlosti z gyroskopu pro omezení potřeby neustálé dostupnosti signálu z globálních navigačních systémů.

Jsou popsány algoritmy pro zpracování dat z šestiosé inerciální jednotky s nehybně umístěnými \ac{MEMS} gyroskopy a akcelerometry, což umožňuje značnou miniaturizaci a snížení ceny, v porovnání s osvědčenými systémy využívající velké a složité mechanické konstrukce gimbalů a gyroskopů. Přestože se přesnost senzorů stále zlepšuje, i díky malé odchylce měřených dat může rychle narůstat chyba odhadované polohy díky integraci úhlové rychlosti a dvojité integraci lineárního zrychlení. Z tohoto důvodu je jednotka opatřena i magnetometrem a \ac{GNSS} modulem pro možnost fúze dalších dat pro co největší zmenšení chyby, tomuto se ovšem bude věnovat až navazující bakalářská práce.

Jsou také věnovány kapitoly samotnému návrhu inerciální jednotky, minimálním požadavkům na komunikace a jejich počtům a rychlostem, způsobu zaznamenávání dat a celkovému blokovému konceptu. V poslední kapitole je také popsán návrh desky plošných spojů inerciální jednotky. 