\chapter*{Úvod}
\phantomsection
\addcontentsline{toc}{chapter}{Úvod}

Tato práce se zabývá poměrně komplexní problematikou inerciální navigace pro použití uvnitř budov z údajů o lineárním zrychlení z akcelerometru a úhlové rychlosti z gyroskopu pro omezení potřeby neustálé dostupnosti signálu z globálních navigačních systémů.

Jsou popsány algoritmy pro zpracování dat z šestiosé inerciální jednotky s nehybně umístěnými mikro-elektromechanickými systémy (\emph{Micro-ElectroMechanical Systems}, \acsu{MEMS}) gyroskopy a akcelerometry, což umožňuje značnou miniaturizaci a snížení ceny, v porovnání s osvědčenými systémy využívající velké a složité mechanické konstrukce gimbalů a gyroskopů. Přestože se přesnost senzorů stále zlepšuje, i díky malé odchylce měřených dat může rychle narůstat chyba odhadované polohy díky integraci úhlové rychlosti a dvojité integraci lineárního zrychlení. Z tohoto důvodu je jednotka opatřena i magnetometrem a globálním družicovým polohovým systémem (\emph{Global navigation satellite system}, \acsu{GNSS}) modulem pro možnost fúze dalších dat pro co největší zmenšení chyby.

Jsou také věnovány kapitoly samotnému návrhu inerciální jednotky, minimálním požadavkům na komunikace a jejich počtům a rychlostem, způsobu zaznamenávání dat a celkovému blokovému konceptu. Je popsán postup konstrukce zařízení a jeho oživení.

Další kapitoly se věnují vývoji firmwaru pro procesor z řady STM32 a jednotlivým funkcím stavových automatů vláken operačního systému reálného času. Jednotka umožňuje přenos naměřených dat do počítače, kde se pomocí vytvořeného skriptu převedou do čitelné podoby.

Předposlední kapitola je věnována zpracování dat, popisuje problémy nepřesnosti senzorů a úspěšnosti navržené kalibrační procedury. Je také otestována fúze dat z jiných senzorů. Poslední část práce je věnována návrhu využití zařízení v laboratorní úloze.