\chapter{Teoretická část studentské práce}
\section{Využití inerciální navigace}
Možnost navigace a znalost polohy je pro lidstvo již dlouhou dobu důležitá jak v průmyslu, tak i v každodenním životě. Pravděpodobně nejrozšířenějším druhem navigace jsou tzv. globální navigační systémy, například GPS. Ovšem pro některé aplikace nemusí být použití GNSS, ať už z politických, či technických důvodů ideální. Pro navigaci v oblasti letectví a námořnictví se začala inerciální navigace využívat kolem roku 1960 a je stále využívána dodnes. \cite{Tittertonc2004}

Díky stále přesnějším a levnějším inerciálním senzorům se rozšiřují možnosti využití inerciální navigace i do běžných průmyslových aplikacích, například v oblastech robotiky, automobilové techniky, nebo i pro údržbu podzemních infrastruktur, mapování kanalizací a další. \cite{Tittertonc2004} Tato práce se zabývá využitím inerciální navigace pro účely určení polohy uvnitř budov, kde je pokrytí signálem globálních navigačních systémů velmi slabé, nebo žádné.

\section{Princip fungování inerciální navigace}
Inerciální navigační systémy pracují na principu nepřímého měření z dat, které poskytuje akcelerometr a gyroskop.   
Akcelerometry poskytují informaci o lineárním zrychlení v prostoru pomocí měření síly $ F $ na definovanou jednotku hmotnosti $ m $ a pomocí 2. Newtonova zákona určí zrychlení $ a $ \cite{Tittertonc2004}
$$ a=\frac{F}{m} $$
Síla $ F $ představuje síly působící na senzor vůči jeho tělu ve volném pádu, skládá se tedy ze statické (tíhové) a dynamické síly způsobené zrychlením vůči Zemi. \cite{Tittertonc2004}
Z tohoto důvodu, pokud je akcelerometr v klidu na povrchu Země, změří zrychlení o velikosti zhruba \SI{9,81}{\meter\per\second\squared}.

Akcelerometry zpravidla měří hodnoty lineárního zrychlení ve třech navzájem pravoúhlých osách. Znalostí počáteční rychlosti $ v(t_{0}) $ a dráhy $ x(t_{0}) $ v čase $ t_{0} $ můžeme pomocí zrychlení $ a $ v časech $ s>t_{0} $ určit rychlost $ v(t) $ a následně dráhu $ x(t) $ pomocí dvou integrací \cite{Grewal2013}
$$ v(t)=v(t_{0}) + \int_{t_{0}}^{t} a(s) \,ds\ $$
$$ x(t)=x(t_{0}) + \int_{t_{0}}^{t} v(s) \,ds\ $$

Aby bylo možné s inerciální jednotkou volně pohybovat v prostoru, je potřeba kromě znalosti dráhy měřit, nebo kompenzovat její natočení. 
Jednou z možností jak kompenzovat rotaci jednotky je připevnění akcelerometrů na gimbal, který bude udržovat jejich natočení vůči zemi konstantní. Tohoto principu se často využívá v letectví, zejména kvůli jejich vysoké přesnosti, ovšem velkou nevýhodou bývá mechanická složitost a velikost. \cite{Grewal2013} \cite{Polak2018}

Druhou možností, jak kompenzovat natočení je měřit jeho úhel a následně zrychlení z akcelerometru rotovat vůči referenčnímu systému.\cite{Grewal2013} \cite{Polak2018}
K tomuto účelu slouží gyroskopy, které měří úhlovou rychlost $ \omega $ otáčení jednotky kolem osy. Podobně jako se zrychlením u akcelerometru, znalostí počátečního úhlu $ \varphi (t_{0}) $ v čase $ t_{0} $ můžeme pomocí úhlové rychlosti $ \omega $ v časech $ s>t_{0} $ určit úhel natočení $ \varphi (t) $, ovšem tentokrát pouze jednou integrací.
$$ \varphi (t)=\varphi (t_{0}) + \int_{t_{0}}^{t} \omega (s) \,ds\ $$

Díky tomu můžou být gyroskopy a akcelerometry nepohyblivě připevněny na mechanickou konstrukci. Jde o tzv. „Strapdown“ typ inerciální navigace.

\subsection{Zavedení vztažných soustav}
Pro účely přehlednosti a exaktnosti bývá v oblastech inerciální navigace zavedeno několik kartézských vztažných soustav. Každá soustava je ortogonální a pravotočivá. \cite{Pekarek2020} \cite{Tittertonc2004} 

\begin{itemize}
\item Inertial frame (i-frame) má počátek ve středu Země. Její osy jsou pevné vůči nepohybujícím se hvězdám.
\item Earth frame (e-frame) má také počátek ve středu Země, její osy jsou pevně vztažené vůči Zemi, tedy rotují kolem i-frame.
\item Navigation frame (n-frame) má počátek ve výchozím bodě navigace. Osy jsou natočené ve směrech sever, východ a vertikálně dolů (North, East, Down).
\item Body frame (b-frame) má počátek v inerciální jednotce a její osy jsou natočené ve směrech náklonu, stáčení a vychýlení jednotky.č
\end{itemize}

S následnými měřenými a vypočtenými daty je často manipulováno jako s vektorem označeným indexem odpovídajícím soustavě, ke které jsou vztaženy (i,~e,~n,~b).

\section{Prv}




